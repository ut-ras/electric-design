\documentclass{standalone}
\usepackage{tikz}
\usepackage[american,cuteinductors,smartlabels]{circuitikz}

\def\dpdt (#1)#2#3#4{
  \begin{scope}[shift={(#1)}]
    %draw inductor; label #2
    \draw (-0.5,-0.5) coordinate (#2+) to[short] (0,-0.5) to[short] (0,0)
    to[L] (0,1)  to[short] (0,1.5) to[short] (-0.5,1.5) coordinate (#2-) to[open] (-0.5,-0.5) ;
    \draw (0.3,-0.2) to[short] (0.3,1.2); 
    \draw (0.4,-0.2) to[short] (0.4,1.2); 
    %draw switch one; label #3
    \draw (1.5,-0.5) coordinate (#3-in) to[short] (1.5,0) to[short, o-o] (1,1) to[short] (1,1.5) coordinate (#3-nc);
    \draw (1.5,-0.5) to[short] (1.5,0) to[open, o-o]  (2,1) to[short] (2,1.5) coordinate (#3-no);
    %draw switch two; label #4
    \draw (3.5,-0.5)  coordinate (#4-in) to[short] (3.5,0) to[short, o-o] (3,1) to[short] (3,1.5) coordinate (#4-nc);
    \draw (3.5,-0.5) to[short] (3.5,0) to[open, o-o]  (4,1) to[short] (4,1.5) coordinate (#4-no);
    \draw[dashed] (0.4,0.5) to (4.5,0.5);
  \end{scope}
}

\def\bat (#1)#2{
  \begin{scope}[shift={(#1)}]
    \draw (0,0) coordinate (#2+) to (0.75,0) to[open] (1.25,0) to (2,0) coordinate (#2-);
    \draw (0.75,0.2) to (0.75,-0.2);
    \draw (0.90,0.1) to (0.90,-0.1);
    \draw (1.10,0.2) to (1.10,-0.2);
    \draw (1.25,0.1) to (1.25,-0.1);
    \draw (0,0) to[open, v=$#2$] (2,0);
  \end{scope}
}

\begin{document}
\begin{circuitikz}
  % manually push out edges TODO: find a better way to do this
  \draw (-5.5,-14) rectangle (7,15.5);
  % the 24V source for everything
  \draw (-2,-0.5) to ++(-1,0) to[open, o-o, v^=$24V source$]  ++(0,2) to ++(1.25,0) ;
  \draw (-2,-12.5) to (-2,15);
  \draw (-1.75,-10.5) to (-1.75,14.75);

  % five relays
  \dpdt(0,-12){R_1}{S1}{S2}
  \dpdt(0,-6){R_2}{S3}{S4}
  \dpdt(0,0){R_3}{S5}{S6}
  \dpdt(0,6){R_4}{S7}{S8}
  \dpdt(0,12){R_5}{S9}{S10}
  % connect them to power
  \draw (R_1+) to ++(-1.5,0);
  \draw (R_2+) to[short, -o] ++(-1.5,0);
  \draw (R_3+) to[short, -o] ++(-1.5,0);
  \draw (R_4+) to[short, -o] ++(-1.5,0);
  \draw (R_5+) to[short, -o] ++(-1.5,0);
  \draw (R_1-) to ++(-1.25,0);
  \draw (R_2-) to[short, -o] ++(-1.25,0);
  \draw (R_3-) to[short, -o] ++(-1.25,0);
  \draw (R_4-) to[short, -o] ++(-1.25,0);
  \draw (R_5-) to[short, -o] ++(-1.25,0);

  % four batteries Hoka means everything else
  \bat(1.5,-12.5){MotorB_1}
  \bat(1.5,-6.5){MotorB_2}
  \bat(1.5,-0.5){HokaiB_1}
  \bat(1.5,5.5){HokaB_2}

  % four chargers, one for each battery
  \draw (S1-no) to ++(0,1.75) to ++(2,0) to[open, o-o, v^=$Charger_1$]  (S2-no);
  \draw (S3-no) to ++(0,1.75) to ++(2,0) to[open, o-o, v^=$Charger_2$]  (S4-no);
  \draw (S5-no) to ++(0,0.75) to[open, o-o, v^=$Charger_3$] ++(2,0) to (S6-no);
  \draw (S7-no) to ++(0,0.75) to[open, o-o, v^=$Charger_4$] ++(2,0) to (S8-no);
  
  % two motor driver connections
  \draw (S1-nc) to[open, o-o, v^=$Motor_1$] ++(0,2) to ++(2,0) to (S2-nc);
  \draw (S3-nc) to[open, o-o, v^=$Motor_2$] ++(0,2) to ++(2,0) to (S4-nc);

  % hook up Hoka 1 to Hoka 2 in series when in run-mode (nc)
  \draw (S5-nc) to ++(0,1.75) to ++(3.75,0) to ++(0,4.25) to (S8-nc);

  % name the output
  \draw (S9-in) to[open, o-o, v=$to Regs$] (S10-in);

  % connect 24V source to charge-mode (no) output relay (R5)
  \draw (-2,15) to ++(4,0) to (S9-no);
  \draw (-1.75,14.75) to ++(5.75,0) to (S10-no);

  % connect Hoka batteries to output relay (R5) in run-mode (nc)
  \draw (S7-nc) to ++(0,1.75) to ++(3.75,0) to ++(0,4.5) to ++(-3.75,0) to (S9-nc);
  \draw (S6-nc) to ++(2,0) to ++(0,12.5) to ++(-2,0) to (S10-nc);
\end{circuitikz}

\end{document}
