\documentclass[master.tex]{subfiles}
\begin{document}
\subfile{power-distribution-diagram}
\subsection{Power Distribution System}
The power system on Rasputin was designed to make it easy to charge and to isolate the sensitive electronics, like the GPS, from the power-line noise caused by driving motors.

Our charging circuit uses five double pull double throw relays to switch between `run' mode as normally connected, and `charge' mode as normally open.
In run mode, Rasputin is discharging his batteries and is capable of driving his motors.
In charge mode, Rasputin is charging his batteries and cannot drive.
Rasputin's sensors are powered regardless of mode.
The relays are powered by a AC to DC converter so that they will switch to charging mode when plugged in.
A diagram of the relay system is provided as figure \ref{relay-diagram} and figure \ref{power-distrobution-diagram} is a diagram of the system as a whole.

\subfile{relay-diagram}
Isolation was a much larger challenge on this robot than we had initially anticipated, as the power draw of the motors would affect both the regulators and the data from sensors.
To isolate the electronics from the motors from the regulators, we simply use separate lead-acid batteries for both systems and opto-isolated motor drivers.
We had a very strange problem that manifested itself as a quick disconnect of every device on the USB hub immediately following a press of the motor's emergency stop button.
Upon analysis, we found out that this was EM interference picked up by the exposed USB traces on the Launchpad.
We solved this by placing the Launchpad in a Faraday cage.
\end{document}
