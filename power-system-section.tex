\documentclass[master.tex]{subfiles}
\begin{document}
\begin{figure}
  \subfile{power-distribution-diagram}
  \caption{A high level view of the power system on Rasputin.
    The arc labelled $A$ acts as a select line for run or charge mode, selecting run mode when Rasputin is not plugged into AC power and charge mode when he is.
    Connections in {\color{red} red} are only available when Rasputin is plugged into AC power.
    Connections in {\color{blue} blue} are only available when Rasputin is not plugged into AC power.
    Connections in black are always available.}
  \label{power-distrobution-diagram}
\end{figure}
\subsection{Power Distribution System}
The power system on Rasputin was designed to make it easy to charge and to isolate the sensitive electronics, like the GPS, from the power-line noise caused by driving motors.

Our charging circuit uses five double pull double throw relays to switch between `run' mode as normally connected, and `charge' mode as normally open.
In run mode, Rasputin is discharging his batteries and is capable of driving his motors.
In charge mode, Rasputin is charging his batteries and cannot drive.
Rasputin's sensors are powered regardless of mode.
The relays are powered by a AC to DC converter so that they will switch to charging mode when plugged in.
A diagram of the relay system is provided as figure \ref{relay-diagram} and figure \ref{power-distrobution-diagram} is a diagram of the system as a whole.

To isolate the electronics from the motors, we simply use separate lead-acid batteries for both systems and opto-isolated motor drivers.
\begin{figure}
  \subfile{relay-diagram}
  \caption{The connections to the relays used to switch from running to charging mode}
  \label{relay-diagram}
\end{figure}
\end{document}
