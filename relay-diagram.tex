\documentclass[master.tex]{subfiles}
\begin{document}
\def\dpdt (#1)#2#3#4{
  \begin{scope}[shift={(#1)}]
    %draw inductor; label #2
    \draw (-0.5,-0.5) coordinate (#2+) to[short] (0,-0.5) to[short] (0,0)
    to[L] (0,1)  to[short] (0,1.5) to[short] (-0.5,1.5) coordinate (#2-) to[open] (-0.5,-0.5) ;
    \draw (0.3,-0.2) to[short] (0.3,1.2); 
    \draw (0.4,-0.2) to[short] (0.4,1.2); 
    %draw switch one; label #3
    \draw (1.5,-0.5) coordinate (#3-in) to[short] (1.5,0) to[short, o-] (1.1,1) to[open, -o] (1,1) to[short] (1,1.5) coordinate (#3-nc);
    \draw (1.5,-0.5) to[short] (1.5,0) to[open, o-o]  (2,1) to[short] (2,1.5) coordinate (#3-no);
    %draw switch two; label #4
    \draw (3.5,-0.5)  coordinate (#4-in) to[short] (3.5,0) to[short, o-] (3.1,1) to[open, -o] (3,1) to[short] (3,1.5) coordinate (#4-nc);
    \draw (3.5,-0.5) to[short] (3.5,0) to[open, o-o]  (4,1) to[short] (4,1.5) coordinate (#4-no);
    \draw[dashed] (0.4,0.5) to (4.5,0.5);
  \end{scope}
}
\begin{figure}
  \scalebox{0.6} {
    \begin{circuitikz}
      % the 24V source for everything
      \draw (-2,-0.5) to ++(-1,0) to[open, o-o, v^=$24V source$]  ++(0,2) to ++(1.25,0) ;
      \draw (-2,-12.5) to (-2,15);
      \draw (-1.75,-10.5) to (-1.75,14.75);
      
      % five relays
      \dpdt(0,-12){R_1}{S1}{S2}
      \dpdt(0,-6){R_2}{S3}{S4}
      \dpdt(0,0){R_3}{S5}{S6}
      \dpdt(0,6){R_4}{S7}{S8}
      \dpdt(0,12){R_5}{S9}{S10}
      % connect them to power
      \draw (R_1+) to ++(-1.5,0);
      \draw (R_2+) to[short, -*] ++(-1.5,0);
      \draw (R_3+) to[short, -*] ++(-1.5,0);
      \draw (R_4+) to[short, -*] ++(-1.5,0);
      \draw (R_5+) to[short, -*] ++(-1.5,0);
      \draw (R_1-) to ++(-1.25,0);
      \draw (R_2-) to[short, -*] ++(-1.25,0);
      \draw (R_3-) to[short, -*] ++(-1.25,0);
      \draw (R_4-) to[short, -*] ++(-1.25,0);
      \draw (R_5-) to[short, -*] ++(-1.25,0);
      
      % four batteries Hoka means everything else
      \draw (S2-in) to[battery, v=$MotorB_1$] (S1-in);
      \draw (S4-in) to[battery, v=$MotorB_2$] (S3-in);
      \draw (S6-in) to[battery, v=$HokaB_1$] (S5-in);
      \draw (S8-in) to[battery, v=$HokaB_2$] (S7-in);
      
      % four chargers, one for each battery
      \draw (S1-no) to ++(0,1.75) to ++(2,0) to[open, o-o, v^=$Charger_1$]  (S2-no);
      \draw (S3-no) to ++(0,1.75) to ++(2,0) to[open, o-o, v^=$Charger_2$]  (S4-no);
      \draw (S5-no) to ++(0,1.75) to ++(2,0) to[open, o-o, v^=$Charger_3$]  (S6-no);
      \draw (S7-no) to ++(0,0.75) to[open, o-o, v^=$Charger_4$] ++(2,0) to (S8-no);
      
      % two motor driver connections
      \draw (S1-nc) to[open, o-o, v^=$Motor_1$] ++(0,2) to ++(2,0) to (S2-nc);
      \draw (S3-nc) to[open, o-o, v^=$Motor_2$] ++(0,2) to ++(2,0) to (S4-nc);
      
      % hook up Hoka 1 to Hoka 2 in series when in run-mode (nc)
      \draw (S5-nc) to ++(0,2.25) to ++(3.75,0) to ++(0,3.75) to (S8-nc);
      
      % name the output
      \draw (S9-in) to[open, o-o, v=$to Regs$] (S10-in);
      
      % connect 24V source to charge-mode (no) output relay (R5)
      \draw (-2,15) to ++(4,0) to (S9-no);
      \draw (-1.75,14.75) to ++(5.75,0) to (S10-no);
      
      % connect Hoka batteries to output relay (R5) in run-mode (nc)
      \draw (S7-nc) to ++(0,1.75) to ++(3.75,0) to ++(0,4.5) to ++(-3.75,0) to (S9-nc);
      \draw (S10-nc) to ++(2,0) to[closing switch, l=$Killswitch$] ++(0,-10) to ++(-2,0) to (S6-nc);
    \end{circuitikz}
  }
  \caption{The connections to the relays used to switch from running to charging mode}
  \label{relay-diagram}
\end{figure}
\end{document}
